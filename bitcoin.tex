

\chapter{How Bitcoin actually works}

%
% lede
%
Thousand of articles have been written about Bitcoin.  Many focus on
the Bitcoin soap opera -- "Is it a bubble?", "Who's getting rich?",
and so on.  Many purport to explain Bitcoin, but nearly all skip over
key details.  Even the more technical articles explain simplified
versions of Bitcoin, often missing crucial elements in how Bitcoin
works.

%
% what I do in this article
%
In this article I explain in detail how Bitcoin works.  It's not one
of the many online ``explanatory articls'' which gloss over crucial
points.  SOme of those articles are rather slick; I am yet to see one
which, beneath the slickness, does a really good job of describing
what's going on.  Rather, my aim is to present all the major ideas
going into Bitcoin in a clear, comprehensible way.

%
% What you need as background
%
In my explanation, I'll assume that you're comfortable with
\link{http://en.wikipedia.org/wiki/Public-key_cryptography}{public key
  cryptography}, and with the closely related idea of
\link{XXX}{digital signatures}.  I'll also assume you're familiar with
\link{XXX}{hashing}.  None of this is especially mathematically
difficult.  It can be (and often is) taught in University freshman
mathematics or computer science classes.  The ideas are beautiful, so
if you're not familiar with them, I recommend taking a few hours to
get familiar.  You can learn it from XXX.

%
% this all sounds like hard work
%
Of course, actually understanding the protocol is work.  Wouldn't it
be more fun to quickly gloss over the protocol, to take Bitcoin for
granted, and instead engage in wild speculation about the end of
taxation, etc?  That does seem to be common for many engaged in
discussion of Bitcoin.  I'll confess that this is an instance where I
think there's nothing quite like putting in the hard work.  It opens
up vistas that otherwise are simply inaccessible.

%
% why is all this of interest?
%
Why does Bitcoin matter?

This is
partially of interest because of Bitcoin ain every major idea that
goes into Bitcoin.  As we'll see, these ideas go far beyond Bitcoin,
and can be used to create many new types of financial instrument.


Many people focus on question such as: "Is Bitcoin a bubble?"  "Can I
use Bitcoin to make myself rich?"  "Is it really a currency?" And so
on.

I don't think you can appreciate Bitcoin without understanding how the
protocol works.  Yes, you can pontificate.  But your words will be hot
air.  You'll be commenting on superficial aspects of Bitcoin, and
missing the fundamentals.

\section{Discovering Bitcoin}

I could explain how Bitcoin works in one big idea dump.  If you're
familiar with public key cryptography then it's not hard to understand
the protocol.  You'd read over it a couple of times, think through
some examples, and you'd get it.

But while you would grasp the protocol, your understanding would be
shallow.  You wouldn't thoroughly understand \emph{why} Bitcoin is the
way it is.  It'd be like memorizing a cooking recipe, without
understanding why the recipe is put together the way it is.

I want to go deeper than that.

Instead of one big idea dump, we'll build Bitcoin up in stages.  I'll
start with very simple questions and ideas about how to create a
digital currency.  Then we'll criticise those ideas, and find ways of
improving them.  Through several rounds, we'll rediscover Bitcoin.

Of course, this approach is more work than simply reading the
protocol.  But it's also more informative.  At the end you will
understand why each element of the protocol is the way it is.  Indeed,
not only will you understand Bitcoin more deeply than people who don't
know the protocol, you will, if you work at it, understand Bitcoin
more deeply than many people who \emph{do} know the protocol, but who
perhaps don't quite appreciate why the protocol is the way it is.

\section{Protocol 0: a signed letter of intent}

How can we design a digital currency?  Let's begin by designing a very
simple (and inadequate) digital currency.  We'll call it an
\emph{InfoCoin}, meaning a unit of currency which is pure information.
Then we'll gradually improve it through several iterations.  

The idea is for Alice to be able to transmit a string of bits -- her
InfoCoin -- to Bob, and the net result is the transfer of money from
Alice to Bob:

XXX - image.

This idea has many problems.  You might naturally wonder: how could we
possibly prevent Alice from using the same string of bits over and
over, and thus minting an infinite supply of money for herself?  How
could we prevent Bob or someone else forging such strings of bits,
effectively stealing from Alice?

These and many other problems need to be overcome in order to use
information as currency.

Let's address one of these problems.  Let's find a way to prevent
someone else forging such a string, and thus stealing from Alice.  To
do this, Alice writes down the message "I, Alice, am giving Bob one
dollar".  She then signs the message using her private key and
announces the signed string of bits to the entire world.

Anyone in the world (including Bob) can then use Alice's public key to
verify that Alice really did sign the message "I, Alice, am giving Bob
one dollar".  This establishes that Alice truly intends to give Bob
one dollar. Furthermore, no-one else could have forged the message,
since only Alice has the private key used to sign the message. 

So using this idea establishes both intent on Alice's part, and
unforgeability by anyone else.  But this protocol still has many
problems.

\section{Protocol 1: Using serial numbers to make coins unique}

A problem with Protocol 0 is that Alice could keep sending Bob the
same signed message over and over.  Suppose Alice sent the message "I,
Alice, am giving Bob one InfoCoin" ten times.  Does that mean Alice
sent Bob ten \emph{different} InfoCoins?  Or was her message
accidentally duplicated?  Worse, perhaps she was trying to trick Bob
into believing that she had given him ten different InfoCoins, when
the message only proves to the world that she intends to transfer one
InfoCoin.

What we'd like is a way of making InfoCoins unique.  They need a label
or serial number.  Alice would sign the message "I, Alice, am giving
Bob one InfoCoin, with serial number 8740348".  Then, later, Alice
could sign the message "I, Alice, am giving Bob one InfoCoin, with
serial number 8770431", and Bob (and everyone else) would know that
that this was a \emph{different} InfoCoin that was being transferred.

Again, just as with Protocol 0, there are many problems with this
proposal.  I'll get to one of those problems in a moment.  Before
getting to that, however, I want to briefly mention a general problem
in understanding Bitcoin (or, indeed, any proposal for digital money,
or cryptographic protocol).  The problem is that when you build a
protocol up in stages, as I am doing, at any given time there are many
problems with the protocol -- ways in which people can cheat, forge
money, steal money, and so on.  It's easy to become pre-occupied
tinking about how it fails in this, that, or the other way.  Those
thoughts can become overwhelming, and it becomes difficult to think
clearly about the protocol.

To overcome this problem, I will begin each stage by identifying just
a single problem that our currency has.  We will focus on just that
problem, and ignore all other issues.  And then I will introduce one
simple idea that solves the problem.  Then we'll do this over again,
identifying a problem and a solution, repeating this pattern many
times, as we gradually build up the protocol.

As we go through this process you're certainly going to have many
thoughts about (often glaring) holes in the protocol-of-the-moment.
You'll just have to trust that I will eventually resolve all those
problems!  To build your confidence, I suggest keeping notes as you go
-- if something's bugging you, then pause in your reading, and write
down as clearly as posssible what's bugging you.  Don't worry about
resolving the problem immediately, just keep a record.  When you get
to the end, and understand the entire protocol, revisit your list.

\section{Verifying serial numbers by having a bank}

I didn't say where Alice would get the serial numbers from.  Should
she just make them up out of thin air?  Obviously, that'd be
convenient for her -- she'd have the capacity to print money!  Not
just that, but there's nothing to prevent her from spending a coin
with the same serial number twice.

One way to solve these problems is to introduce a <em>bank</em> which
is the ultimate source of InfoCoins.  This bank provides serial
numbers, keeps track of who has which coins, and verifies that
transactions really are legitimate, preventing Alice from simply
making up serial numbers of her own.

To see how this might work in more detail, let's suppose Alice goes
into the bank, and says "I want to withdraw one InfoCoin from my
account".  The bank reduces her account balance by one InfoCoin, and
gives her a new, never-before used serial number, let's say 1234567.

WHY WHOULD SHE GO TO A BANK IN THE FIRST PLACE?  IT'D BE THE ULTIMATE
ORIGIN OF ALL BITCOINS.

Now, when Alice wants to transfer an InfoCoin to Bob, she'd sign the
message "I, Alice, am giving Bob one InfoCoin, with serial number
1234567".  But Bob doesn't just accept the InfoCoin.  Instead, he
contacts the bank, and verifies that: (a) the InfoCoin with that
serial number belongs to Alice; and (b) Alice hasn't already spent the
InfoCoin.  If both those things are true, then Bob accepts the
InfoCoin, and the bank updates their records to show that the InfoCoin
with that serial number is now in Bob's possession, and no longer
belongs to Alice.





What ideas do I want to add?


XXX: MODIFICATION: Modify by replacing the serial number by a pointer
to the previous transaction.


Make everyone the bank

Problem: We don't want a trusted bank.

The idea is to make it so <em>everyone</em> is the bank.  In
particular, we're going to assume that everyone using InfoCoin keeps a
complete record of which coins belong to which person.  You can think
of this as being like a ledger showing all transactions.  We'll call
it the <em>block chain</em>, since that's what the complete record is
called in Bitcoin.

Now, suppose Alice wants to transfer an InfoCoin to Bob.  She signs
the message "I, Alice, am giving Bob one InfoCoin, with serial number
1234567".  Bob broadcasts out a message to the network, saying: "Hey,
does Alice have InfoCoin 1234567?"  Provided the majority of users in
the network are honest, Bob gets a message back saying "Yes, Alice
does".  Bob accepts the InfoCoin

</div>

<h3>Proof-of-work</h3>

<div class="section">

There's a problem here with honesty.  A thief could imitate a large
number of separate InfoCoin users, and use that to approve fraudulent
transactions.  

For instance, if Bob was the thief, he could steal from Alice by
broadcasting out to the network the message "I, Alice, am giving Bob
one InfoCoin, with serial number 1234567".  Then he 

We can solve this using an idea known as <em>proof-of-work</em>.  The
idea is to make it <em>hard</em> to take part in the approval process.
If people have to expend tremendous amounts of processing power to
corrupt the approval process, then that'll be a real obstacle to them
doing so.

Here's how it works.

Suppose Alice broadcasts to the network the news that "I, Alice, am
giving Bob one InfoCoin, with serial number 1234567".  And suppose
Charlie hears this message.

As other people
hear the message, they don't have the ability to immediately approve
the mesage.  Instead, they add it to a queue of pending transactions
that they've been told about, but which haven't yet been approved by
the network:

XXX

Suppose, for instance, that a third party, Charlie, has a particular
queue of pending transactions:

XXX

Charlie can see that all the transactions are valid &ndash; they all fit
with his existing ledger.  He would <em>like</em> to help out by
broadcasting the validity to the entire network.  But, unfortunately,
the rest of the network doesn't know whether or not to trust Charlie.

So here's what Charlie has to do instead.  He takes his entire list of
pending transactions.  And he chooses a random number as well &ndash;
called a <em>nonce</em>.  Then he hashes the list of transactions
together with the nonce to produce a hash value:

XXX

What Charlie has to do is to find a nonce so that the output hash
starts with a long string of zeroes:

XXX

This is <em>hard</em> to do.  


You can think of it as having a lottery to approve transactions.  But
each ticket in the lottery costs a little bit of computing power.  If
millions of people are entering, then Charlie would only have a very
small chance to corrupt the process, unless he expends enormous (and
very costly) computational resources.

</div>

<h3>Why have a chain?</h3>

<div class="section">

Why have a chain?


Why point 

To approve the 

In particular, they actually have to solve a difficult puzzle before
they can approve a transaction.



when a potential transaction is broadcast



Make it take constant time.


CHEATING: WORK VERY HARD TO 

</div>

<h3>Reward for work, and the origin of InfoCoins</h3>

<p>
It's all very well to ask Charlie to work hard to validate a
transaction.  But why would he bother to do so, expending
computational power &ndash; power that costs him money &ndash; merely to help
validate other people's transactions?
</p>

<p>
The solution to this problem is to reward the people who help find the
proof-of-work.  In particular, suppose we credit whoever validates the
block with some InfoCoins.  Provided the InfoCoin reward is large
enough that gives them an incentive to participate in the mining.
</p>

<p>
Of course, this gives rise to many economic questions.  It changes the
total supply of InfoCoins.  And it sets up interesting economics
around mining.  Does InfoCoin mining end up in concentrated in the
hands of a few, or many?  If it's just a few, doesn't that endanger
the security of the system?  We'll look at these questions below in
the context of Bitcoin.
</p>

<h3>Bitcoin</h3>

Alright, we've now understood the main ideas behind BitCoin.  Let's
move away from InfoCoin, and describe the actual BitCoin protocol.

You install a wallet program on your computer.

Suppose someone wants to buy something from you &ndash; maybe you're a
merchant who has set up an online store, and you allow people to pay
using Bitcoin.

Well, you first tell your wallet to generate a private key / public
key pair.  Then your wallet generates a <em>Bitcoin address</em>: this
is just a hashed version of the public key:

XXX

You then send your Bitcoin address to the person who wants to send you
Bitcoins:

XXX

This is quite safe -- after all, it's a hash of your public key, which
is intended to be known by the world, anyway.  (I'll return later to
the use of the hash.)

They [XXX] then generate a <em>transaction</em>.  Let's take a look at
the raw data from an <a
href="http://blockexplorer.com/tx/7c402505be883276b833d57168a048cfdf306a926484c0b58930f53d89d036f9">actual
transaction</a> transferring $0.31900000$ (XXX) bitcoins.  I've added
line numbers, for ease of reference in the explanation below, but
otherwise this all the data in the transaction.

<pre>
1.  {"hash":"7c402505be883276b833d57168a048cfdf306a926484c0b58930f53d89d036f9",
2.  "ver":1,
3.  "vin_sz":1,
4.  "vout_sz":1,
5.  "lock_time":0,
6.  "size":224,
7.  "in":[
8.    {"prev_out":
9.      {"hash":"2007aec728a1b1dcd36aa476e873926c412e94b2d16d060a7c97014b83a00c3e",
10.      "n":0},
11.    "scriptSig":"304502205014856cdf89da70ad9a4f223bac4e5477da5c6cb69ef2b9f8b5f8548e21307e0221009bfe2698f1eb1c561f41981d8e78c11d9e685a70e682f144ee6c8ab5ecb0497c01 042b2d8def903dd62d0c4161ed8d4ccfa5967e11a28e65cb141235b7c27d8ef6aa3bd63be077323cf3d7e0e8895b264b94feb4b40478b431da6f45dfc8e1004f62"}],
12. "out":[
13.   {"value":"0.31900000",
14.    "scriptPubKey":"OP_DUP OP_HASH160 a7db6ff121871c65a8924b8e40f160d385515ad7 OP_EQUALVERIFY OP_CHECKSIG"}]}
</pre>

<p>
Let's go through this, line by line.
</p>

<p>
Line 1 contains the hash of the remainder of the transaction.  This is
used as an identifier for the transaction.
</p>

<p>
Line 2 tells us that this is a transaction in version 1 of the Bitcoin
protocol.
</p>

<p>
Lines 3 and 4 tell us that the transaction has one input and one
output, respectively.  I'll talk below about transactions with more
inputs and outputs, and why that's useful.
</p>


<p>
Line 5 is a detail.  It contains the value for <code>lock_time</code>,
which can be used to control when a transaction is finalized.  In this
case, as for most Bitcoin transactions being carried out today, the
<code>lock_time</code> is set to 0, which means the transaction is
finalized immediately.
</p>

<p>
Line 6 is also a detail, telling us the size (in bytes) of the
transaction.  The important thing here is to understand that this
<em>isn't</em> the monetary amount being transferred!  That comes
later.
</p>

<p>
Lines 7 through 11 defines the input to the transaction.  In
particular, lines 8 through 10 tell us that the input is to be taken
from the ouput from an earlier transaction, with the given
<code>hash</code>.  The <code>n=0</code> tells us it's to be the first
output from that transaction &ndash; we'll see soon how multiple
outputs (and inputs) from a transaction work, so don't worry too much
about this for now.  Line 11 contains the signature, followed by a
space, and then the public key of the person sending the money.  Note
that the signature is for the data in lines 8 through 10.
</P>

<p>
Lines 12 through 14 define the output from the transaction.  In
particular, line 13 tells us the value of the output, 0.39000000
bitcoins, or 39,000,000 satoshis.  Line 14 is somewhat complicated.
The main thing, however, is that the long string "a7db6ff12..." is the
Bitcoin address of the intended recipient of the funds.
</P>


<h3>Transactions with multiple inputs and outputs</h3>

I described above how a transaction with just a single input and a
single output works.

In practice, it's often extremely convenient to create Bitcoin
transactions with multiple inputs or multiple outputs.

For example, suppose I want to send you else 150 satoshis.  They do so
by spending money from a previous transaction in which they received
200 satoshis.  Of course, they don't want to send the entire 200
satoshis to the other person.  The solution is to send the other
person 150 satoshis, and to send one of their own Bitcoin addresses 50
satoshis.  Those 50 satoshis are known as <em>change</em>.  Of course,
it differs a little from ordinary change, since change is what you pay
yourself.




The second output is defined lines 23 and 24, with a similar format to
the first output.

The sum of all the inputs must be equal to the sum of all the outputs.

If not...XXX.  Transaction fee.


We can visualize the entire transaction content as follows:

XXX



Here's the raw data from an <a
href="http://blockexplorer.com/tx/99383066a5140b35b93e8f84ef1d40fd720cc201d2aa51915b6c33616587b94f">actual
transaction</a>.  I've added line numbers, for ease of reference in
the explanation below, but otherwise this all the data in the
transaction.

<pre>
1. {"hash":"99383066a5140b35b93e8f84ef1d40fd720cc201d2aa51915b6c33616587b94f",
2. "ver":1,
3. "vin_sz":3,
4.  "vout_sz":2,
5.  "lock_time":0,
6.  "size":552,
7.  "in":[
8.    {"prev_out":{
9.      "hash":"3beabcb8818f8331dd8897c2f837a4f6fe5cc5e0f3a7c8806319402d2467c30a",
10.        "n":0},
11.     "scriptSig":"3044022062ea95519d5d91cbce4086a63b8cd509a4900ba59063b69286236527e31a228e022076de59315406b7ec3a7414c98b5d32f47d11b9a786d31cf44883f3fb5812aa4001 04c7d24c58ae83f38bd2fb496758ff544965d58e7e5471ccb7349c8c404c64d0a57b562a20dfdcf152e0a401473ba520e387bf2516a4841a5f5bf5701b6fc09552"},
12.    {"prev_out":{
13.        "hash":"fdae9b76f974a9476f81c52d5ae1fbbd48cb840722e0805e56de1f9d2da0d9bc",
14.        "n":0},
15.      "scriptSig":"304502201c08b87eec72c4cb77369da7ef108ac18f29a67dff8865163cac3b155a0e9bf4022100afd61ce024ed33c4eee5e2f5cbc13203527a3b708f14c9573943132c061f800301 026e15a0c21d5f8c708e8b86d2f57ab1b7d31afee4a479e30af29d705532cf59ce"},
16.    {"prev_out":{
17.        "hash":"20c86b709ff4747866ef9f59788d1e18de81956c6501854a15707ccaa11076ce",
18.        "n":1},
19.      "scriptSig":"3044022038203b996b306916848732679b320be3c511870249da5b03a719f5a1f39cf646022070fd8c34a6ff73ebc8272e5ba717b9d3ef7100846bdbe4049808683d475d478001 038a52383beaf9711915f338f9c063332f39443358c1e4bc942da69551093b0896"}],
20.  "out":[
21.    {"value":"0.01068000",
22.      "scriptPubKey":"OP_DUP OP_HASH160 e8c306229529009d596689cb9212d6519cf6de8a OP_EQUALVERIFY OP_CHECKSIG"},
23.    {"value":"4.00000000",
24.      "scriptPubKey":"OP_DUP OP_HASH160 d644e36b9b295b3a1fa6ca2f816ba1f9340f4806 OP_EQUALVERIFY OP_CHECKSIG"}]}
</pre>

Let's go through this, line by line.

Line 1 contains the hash of the remainder of the transaction.  This is
used as an identifier for the transaction.

Line 2 tells us that this is a transaction in version 1 of the Bitcoin
protocol.

Lines 3 and 4 tell us that the transaction has three inputs and two
outputs, respectively.

Line 5 is a detail.  It contains the value for <code>lock_time</code>,
which can be used to control when a transaction is finalized.  In this
case, as for most transactions today, the <code>lock_time</code> is
set to 0, which means the transaction is finalized immediately.

Line 6 tells us the size of the transaction.

Lines 7 through 19 define a list of the inputs to the transaction.
These are outputs from previous Bitcoin transactions.

The first input is defined in lines 8 through 11.  

In particular, lines 8 through 10 tell us that the input is to be
taken from the <code>n=0</code>th output from an earlier transaction,
with the given <code>hash</code>.  Line 11 contains the signature,
followed by a space, and then the public key of the person sending the
money.  Note that the signature is for the data in lines 8 through 10.

Lines 12 through 15 define the second input, with a similar format to
lines 8 through 11.  And lines 16 through 19 define the third input.

Lines 20 through 24 define a list containing the two outputs from the
transaction.

The first output is defined in lines 21 and 22.  Line 21 tells us the
value of the output, 0.01068000 bitcoins, or 1,068,000 satoshis.  Line
22 is somewhat complicated.  The main thing, however, is that the long
string "e8c30622..." is the Bitcoin address of the 

The second output is defined lines 23 and 24, with a similar format to
the first output.

The sum of all the inputs must be equal to the sum of all the outputs.

If not...XXX.  Transaction fee.

Suppose someone wants to send someone else 150 satoshis.  They do so
by spending money from a previous transaction in which they received
200 satoshis.  Of course, they don't want to send the entire 200
satoshis to the other person.  The solution is to send the other
person 150 satoshis, and to send one of their own Bitcoin addresses 50
satoshis.  Those 50 satoshis are known as <em>change</em>.  Of course,
it differs a little from ordinary change, since change is what you pay
yourself.

We can visualize the entire transaction content as follows:

XXX



Change.

<h3>Oddities of Bitcoin</h3>

Bitcoin has several oddities.

One oddity is the use of addresses.  Why not use the public key as the
address, rather than a hash of the public key?

This seems to be a fairly arbitrary design decision.  

One seemingly plausible explanation is that it's to add extra
security.  Even if the public key cryptosystem were broken, having
only hashed public keys public would still provide some level of
security.

Unfortunately, this story isn't especially plausible.  It's true that
to receive money, you only need reveal your Bitcoin address.  But to
spend money, you need to reveal your public key, since it's used to
sign the transaction.  Once you've revealed your public key, people
can take the hash to recover your address, and to see what previous
transactions it's participated in, as a recipient of funds.

In other words, your public key is kept secret


There is one scenario in which the use of addresses makes sense.
Suppose that the public key cryptography apocalypse occurs -- someone
discovers how to crack the public key cryptosystem.  Then everyone
whose public keys are known would immediately have their account
entirely compromised.  

Now, suppose that people never used the same Bitcoin address twice to
receive funds.

So provided you never use the same Bitcoin address twice

It's pretty common advice to  

A second oddity is that the transaction as a whole isn't signed.  XXX
&ndash; TRANSACTION MALLEABILITY.




The transaction is labelled by a hash.  This is a hash of parts of the
remainder of the transaction.  It's not completely clear to me which
parts &ndash; apparently the transaction is somewhat malleable because of
this, but the amounts being paid out, senders and recipients can't be
changed.

A transaction actually doesn't have just one sender and receiver, but
can potentially involve many parties. 


Number of inputs.

Number of outputs.

Each input contains: (1) The hash of (part of) a previous transaction;
(2) an index for an output in that transaction; (3) the public key of
the sender; and (4) a signature for (part of) the transaction.



Each output contains: (1) The value for that output; (2) The hashed
public key of the intended recipient.

Note that the sum of the output values must equal the sum of the
inputs.

Why does it contain the hashed public key of the intended recipient?
I don't really quite get this.


Comment on the author: The paper is not especially well written.  It
contains a number of vague statements, and some <em>non
sequiturs</em>.  It looks to me most likely to be the work of someone
very bright, and very well informed, but probably not a long-time
professional.  It wouldn't surprise, though, if they were a graduate
student in the field, or had equivalent background.




How does Bitcoin relate to karma systems?


The dull questions:

+

The privacy premium.




Good questions: When it is economic to invest in Bitcoins?

What is the impact of having a fixed supply of money?


The economics of privacy.  

Will people pay a privacy premium?




This results in the growth of



What problem is proof of work solving?


The supply of money.  The Bitcoin addresses responsible for validating
the block receives a reward.  This reward is carried out as the first
transaction in the block.




Kickstarter at the protocol level


Generalized eminent domain


Anonymity?  Probably never.  The problem is that too much of the graph
is known.  Can always de-anonymize.  Might we hide steganographically?
Idea is to create a large number of transactions which swamp the
signal from the small number.

DDoS the Bitcoin network.  Simply start creating a very large number
of transactions.  This will overcome the network's ability to verify.
It wouldn't take a very large amount of power.



<h3>Problem: concentration of Bitcoin mining power in the hands of a
few</h3>

Consider what happens over time.  There's a good incentive to spend
more and more money.

Well, what happens at that point?  

It depends.  Suppose someone can apply ingenuity 

the problem is that, eventually, you'd expect
the cost of mining to .  You'd actually expect the number of pople

People switch their mining setups on and off, depending on the current
exchange rate.

You see this happening already.  

Rushes occur.


It's conventional to use uppercase Bitcoin when talking about general
aspects of Bitcoin.  So we'll talk about the Bitcoin protocol, the
Bitcoin foundation, and so on.  And to use lowercase bitcoin when
referring to specific denominations, as in "She sent 5 bitcoins to pay
for her order."

The name "Bitcoin" makes it sound as though a bitcoin is a
diminunitive unit of currency.  In fact, as I write a single bitcoin
trades for about 200 US dollars, and the exchange rate has in the past
gone nearly as high as 240 US dollars.  If Bitcoin were ever to go
truly mainstream I have little doubt the exchange rate would go much
higher.  And so a bitcoin isn't diminutive at all, it's big dollars.

In actual fact, the bitcoin isn't the basic unit of currency.  A
single bitcoin can be split up into 100 million "satoshis", named
after the originator of Bitcoin,
\link{https://en.bitcoin.it/wiki/Satoshi_Nakamoto}{Satoshi Nakamoto}.
If 200 US dollars is one bitcoin, then a satoshi is 1/500th of a US
cent.  

Bitcoin is the Wild West.  There's all sorts of shady behaviour and
hucksters looking to make a quick buck.  It's also true that, as in
the gold rush, it may be better to make money selling shovels than
digging for gold.  One oddity is that you'll read a lot of online
accounts by people who seem very knowledgeable and utterly confident
-- and who turn out to be completely wrong.  Sometimes this is
indicative of shady behaviour.  But more often it's just routine
over-confidence.  Don't believe everything you read.

\end{document}